\documentclass{article}
\usepackage[utf8]{inputenc}
\usepackage[italian]{babel}
\usepackage[T1]{fontenc}

\title{Galileo Galilei \\
\large Relazione del progetto per l'insegnamento di Algoritmi e Strutture di Dati}
\author{
  Mattia Girolimetto (fill),
  Luca Tagliavini (0000971133)
}
\date{
	Universit\`a di Bologna \\
  \today
}

\begin{document}

\maketitle

\section{Problema computazionale}

Lo scopo del progetto \`e quello di implementare un algoritmo efficiente e ottimale
volto alla ricerca delle mosse migliori in un gico $(m,n,k)$ dove si devono
allineare $k$ simboli in una griglia $m \times n$.

A questo fine viene sfruttata una variazione dell'algoritmo MiniMax con potatura
AlphaBeta denominata \emph{Principal Variation Search}~\cite{negascout}. Esso
sfrutta una ricerca limitata in profondit\`a e garantisce di analizzare solo e soltanto
i nodi che sarebbero stati analizzati da AlphaBeta, spendendo meno tempo su
quelli che sappiamo essere meno profiqui grazie ad una \emph{ricerca con finestra $\alpha$-$\beta$ nulla}.
Per poter sfruttare questa ottimizzazione si necessita della possibilit\`a di ordinare i
figli di un dato stato di gioco, che pu\`o essere ottenuta tramite la tecnica
dell'\emph{Iterative Deepening}~\cite{id}. Oltretutto in questo modo si visita al
massimo l'albero di gioco entro i limiti di tempo.

Svariate combinazioni di mosse possono portare a stati gi\`a analizzati, i quali
vengono mantenuti dentro una cache per evitare di valutarli pi\`u volte.
La stima delle griglie non foglia viene gestita da una componente euristica che
tiene in considerazione il numero di serie di ogni giocatore, la loro lunghezza
e la vicinanza ad altre celle libere.

\section{Scelte Progettuali}

\subsection{Select Cell}
\subsection{Cache}
\subsection{PVS}
\subsection{Valutazione Euristica}

\pagebreak
\bibliographystyle{ieeetr}
\bibliography{report}

\end{document}
