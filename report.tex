\documentclass{article}
\usepackage[utf8]{inputenc}
\usepackage[italian]{babel}
\usepackage[T1]{fontenc}

\title{Galileo Galilei \\
\large Relazione del progetto per l'insegnamento di Algoritmi e Strutture di Dati}
\author{
  Mattia Girolimetto (fill),
  Luca Tagliavini (0000971133)
}
\date{
	Universit\`a di Bologna \\
  \today
}

\begin{document}

\maketitle

\section{Problema computazionale}

Lo scopo del progetto \`e quello di implementare un algoritmo efficiente e ottimale
volto alla ricerca delle mosse migliori in un gico $(m,n,k)$ dove si devono
allineare $k$ simboli in una griglia $m \times n$.

Si fa uso di una variante dell'algoritmo MiniMax con potatura
AlphaBeta denominata \emph{Principal Variation Search}~\cite{negascout}. Viene
svolta una ricerca limitata in profondit\`a analoga a minimax, espandendo
interamente i nodi denominati \emph{Principal Variation} e parzialmente quelli
meno interessanti. Applicando un \emph{Iterative Deepening}~\cite{id} abbiamo
a disposizione un ordinamento dei sottoalberi basato sui valori euristici delle
ricerche precedenti e possiamo raggiungere la profondit\`a massima nei limiti
imposti.

Svariate combinazioni di mosse possono portare a stati gi\`a analizzati, i quali
vengono mantenuti dentro una cache non valutarli pi\`u volte.
La stima dei stati di gioco non finali viene gestita da una componente euristica che
tiene in considerazione il numero di serie di ogni giocatore, la loro lunghezza
e la vicinanza ad altre celle libere.

\section{Scelte Progettuali}

\subsection{Select Cell e Iterative Deepening}

Dopo aver aggiornato la copia locale del tavolo di gioco si va alla ricerca
della migliore cella da giocare, tramite Iterative Deepening.
In questo modo si eseguono ricerche MiniMax con profondit\`a sempre maggiore
e si restituisce il risultato della ricerca pi\`u recente. Questa tecnica non
peggiora il costo asintotico in quanto il solo peso di MiniMax alla
massima profondit\`a assorbe quello di tutte le ricerche precedenti.
Tuttavia aumenta sicuramente il numero di operazioni svolte dal calcolatore:
per minimizzare questo danno si ricorre alla Principal Variation Search.

\subsection{Cache}
\subsection{PVS}
\subsection{Valutazione Euristica}

\pagebreak
\bibliographystyle{ieeetr}
\bibliography{report}

\end{document}
